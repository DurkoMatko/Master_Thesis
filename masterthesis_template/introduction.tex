With the social media boom in the last decade, amount of generated data increased exponentially. Current average daily tweets count oscilates around 500 million\footnote{https://www.omnicoreagency.com/twitter-statistics/}. These data contain valuable information of all kinds and that is the reason why with social media rise also natural language processing and sentiment analysis became hot research areas. They provide insight into data created by real people, users and consumers. More and more businesses start to take advantage of this and move from traditional means of data gathering and analysis to this new and still rising space. Reviews, blogs, statuses, tweets - these all provide rich environment to extract knowledge from.
Sentiment analysis has several approaches such as classification, regression or ranking and each is applied to tackle different tasks. Probably the most intuitive one is classification which assigns one of defined classes to each analysed textual document. That is also an approach I have decided to use in this thesis. \\
\\
I have analysed tweets of several open-source projects (OSS) and tried to find a relationship between their release frequency and sentiment. Analysing tweets differs to another sentiment analysis implementations and represents interesting challenge as tweets are special type of textual data limited to 280 characters per "document" while often using very specific language.\\
\\
Another growing field of natural text processing is text semantics interpretation, topic modelling and its implementation within bug tracking systems either for filtering or flagging duplicate reported bugs. As as second part of my thesis, I have done somewhat similar process - I have tried to pair social media discussions with their respective reported Git issues based on similarity of their textual features. Here I have used a common approach of transforming documents into vectors and computing their cosine similarity.

\section{Motivation}
Motivation behind this thesis lies in my interest in machine learning and the fact that I unfortunately did not manage to explore this field over the course of my studies. After taking some very basic Coursera\footnote{https://www.coursera.org/learn/machine-learning} courses in this area, I felt I wanted to learn more and rather than finishing some online course exercises, I wanted to implement the whole workflow of such algorithm from the ground up and experience all the nitty-gritty myself. Machine learning, sentiment analysis and data mining in general are fields on the rise and it is always good to have as wide knowledge as possible. That is also the reason why my thesis is relatively wide-spread and targets many areas of data mining. I can very easily get drawn from one interest to another and that is also a case in this thesis, just in a smaller scale. I am also a fan of open-source movement and mindset and I liked the idea of combining these two into my thesis.