My goal of pairing bugs with social media might be somewhat similar to nowadays very active field of bug report duplicate discovery but has also a lot common with topic modeling and general text similarity algorithms.

To link any two texts based on their content, there's an obvious need for understanding what the text features are. 
To do this, there are several ways how to calculate text (string) similarity value or one can even execute so called "Topic modeling" algorithm and try to connect documents based on their matching topics.

Topic modeling is a method used to organize and summarize large textual information. It's used to discover hidden topical patterns and annotate documents according to these topics. It can also be described as a method of finding group of words (i.e topic) from in a text that best represents the information in the collection.

The obstacle of using a topic modeling for my case is that neither SO nor Reddit questions are not long enough. This could potentially be avoided by concatenating the whole discussions into one long text but these are still too topic-specific to get reasonable output. In this case, topic modeling output is not granular enough to differentiate among similar texts which are all from the very same domain. 

Because the output provided by topic modeling wasn't enough to pair particular items, I searched for, found and considered several alternative approaches.

\section{Aproaches}
There are many ways and approaches how to find out whether 2 texts are similar or not. Most of them are to some extent very similar as the general need says some textual features of need to be extracted and compared.

I've tried following algorithms:
\begin{enumerate}
\item Nouns extraction
\item String similarity using NLTK
\item String similarity using Gensim
\item The textual properties, title and description
are concatenated together and then used to numeric features using the simple TakeLab system. This approach was shown by Lazar, Ritchey and Sharif. \cite{lazar2014improving}
\end{enumerate}


\section{Available data}
\paragraph{Git issue reports:}
Using the approach described in the section \ref{ssec:issuesMining}, I've downloaded 96,651 issue reports from Git while 25,978 of those labeled as bug or similar. Because this part of the thesis was more just a PoC than the part with sentiment analysis, I've decided not to work with all the data and rather just picked several projects of interest. The bug counts among this projects is displayed in the table \ref{table:projectIssuesDistribution}.


\begin{table}[H]
\centering
\begin{tabular}{ |p{3cm}||p{3cm}|}
 \hline
\textbf{ Framework }& \textbf{Bug count}\\
 \hline
 NodeJS   & 1615\\ \hline
 AngularJS &   2225 \\ \hline
 EmberJS & 1284\\ \hline
 VueJS & 353\\ \hline
 Aurelia & 73\\ \hline
 Bower & 155\\ \hline
\end{tabular}
\caption{Bug count per project}
\label{table:projectIssuesDistribution}
\end{table}

\paragraph{Stack overflow questions:}
SO mining has been described in section \ref{ssec:GettingData} and I've downloaded 5,847 questions. There are thousand questions for AngularJS, NodeJS, Bower, Ruby on Rails and VueJS each and EmberJS has only 847 questions. Downside is, that despite having a lot of questions, it doesn't necessarily mean that each and every one of them talks about some known bug. Actually, opposite is true as out of all those questions only very tiny percentage does (AngularJS - 1, NodeJS - 2, EmberJS - 2, VueJS - 0).

\paragraph{Reddit dialogues:}Reddit subreddits mining has been described in the same subsection as SO mining. Results of this process are shown in the table \ref{table:redditDiscussionsDistribution}.


\begin{table}[H]
\centering
\begin{tabular}{ |p{3cm}||p{3cm}|}
 \hline
\textbf{ Framework }& \textbf{Submissions count}\\
 \hline
 NodeJS   &  108\\ \hline
 AngularJS &   43 \\ \hline
 VueJS & 20\\ \hline
 EmberJS & 13\\ \hline
\end{tabular}
\caption{Reddit submissions counts}
\label{table:redditDiscussionsDistribution}
\end{table}


\section{Similarity results}
Linking items and in general finding similarity among these already area-specific texts proved to be a problematic task. 

\subsection{NLTK}
\paragraph{Stack Overflow:}
The average similarity between SO questions talking about particular issue and that particular issue description is 0.316 without body preprocessing and 0.292 with body preprocessing.

The distribution of similarities in buckets by increased by 0.05 can be seen in histogram in Figure \ref{fig:GitStackMatchesHistogram}

\begin{figure}[H]%
    \centering
	\includegraphics[width=8cm]{gitStackMatchesHistogram.jpg}
    \caption{Histogram of similarities distribution among git issues and their matching SO questions}%
    \label{fig:GitStackMatchesHistogram}%
\end{figure}

For random SO questions, amount of comparisons to Git issues needed to be limited. If every SO question would be compared to every Git issue, time of computation would exceed timeframe of this thesis. Every SO question was therefore compared to 3 random git issues and resulting average similarity scores are in following figures.

\begin{figure}[H]%
    \centering
	\includegraphics[width=8cm]{AureliaStackWithRandom3Bugs.jpg}
    \caption{Histogram of Aurelia SO questions and random git issues. Average similarity was 0.244}%
    \label{fig:AureliaStackWithRandom3Bugs}%
\end{figure}


\begin{figure}[H]%
    \centering
    \subfloat[EmberJS average similarity - 0.247]{{\includegraphics[width=6cm]{EmberStackWithRandom3Bugs.jpg} }}%
    \qquad
    \subfloat[Bower average similarity - 0.217]{{\includegraphics[width=6cm]{BowerStackWithRandom3Bugs.jpg} }}%
    \caption{EmberJS and Bower similarity histogram}%
    \label{fig:BowerEmberWithRandom3Bugs}%
\end{figure}

\begin{figure}[H]%
    \centering
    \subfloat[VueJS average similarity - 0.255]{{\includegraphics[width=6cm]{VueJSStackWithRandom3Bugs.jpg} }}%
    \qquad
    \subfloat[AngularJS average similarity - 0.258]{{\includegraphics[width=6cm]{AngularStackWithRandom3Bugs.jpg} }}%
    \caption{VueJS and AngularJS similarity histogram}%
    \label{fig:VueAngularWithRandom3Bugs}%
\end{figure}


Comparing Git bugs descriptions with SO questions talking about own project issues and general issues, similarity values are following \ref{table:StackOverflowNLTKsimilarity}. Same table but for Reddit discussions can be seen in \ref{table:RedditNLTKsimilarity}

\begin{table}[H]
\centering
\begin{tabular}{ |p{3cm}||p{5cm}|p{5cm}|}
 \hline
\textbf{ Framework }& \textbf{Own issues similarity}& \textbf{All issues similarity}\\
 \hline
 NodeJS   & 0.265 & X\\ \hline
 Angular & 0.241 & X\\ \hline
 EmberJS & 0.282 & X\\ \hline 
 VueJS &   0.261 & X\\ \hline
\end{tabular}
\caption{NLTK similarity values for SO questions}
\label{table:StackOverflowNLTKsimilarity}
\end{table}

\paragraph{Reddit:}
Here I've calculated the similarity between the bug description and either particular comment in the reddit discussion which mentioned the bug or the whole discussion itself. Average similarity score for all considered projects (NodeJS, AngularJS, VueJS and EmberJS) was 0.481 for the whole discussion and 0.396 for the comment itself. Detailed scores for each project can be found in table \ref{table:RedditNLTKsimilarity}. Subreddit for EmberJS didn't reference any of its own bugs.

\begin{table}[H]
\centering
\begin{tabular}{ |p{3cm}||p{3cm}|p{4cm}|}
 \hline
\textbf{ Framework }& \textbf{Bug comment}& \textbf{Whole discussion}\\
 \hline
 NodeJS   & 0.447 & 0.507\\ \hline 
 AngularJS & 0.306 & 0.57 \\ \hline 
 VueJS &   0.359 & 0.380\\ \hline
\end{tabular}
\caption{Reddit NLTK similarity values}
\label{table:RedditNLTKsimilarity}
\end{table}

This indicates that the semantic meaning of the bug is better expressed in the whole discussion rather than just the particular comment which referenced the bug. This made me question if it could be generalized that longer the text is, more similar it is to actual bug description. I've plotted a relationship between similarity score and length in Figures \ref{fig:SimilarityLengthRelationshipComment} and \ref{fig:SimilarityLengthRelationshipDiscussion}.

\begin{figure}[H]%
    \centering
	\includegraphics[width=8cm]{SimilarityLengthRelationshipDiscussion.jpg}
    \caption{Discussion lengths and similarity scores with the issue}%
    \label{fig:SimilarityLengthRelationshipDiscussion}%
\end{figure}

\begin{figure}[H]%
    \centering
	\includegraphics[width=8cm]{SimilarityLengthRelationshipComment.jpg}
    \caption{Comment lengths and similarity scores with the issue}%
    \label{fig:SimilarityLengthRelationshipComment}%
\end{figure}




\section{Using GIT labels}
One more considered approach how to recognize an issue's topic was using the GIT labels. Labels on GitHub help you organize and prioritize your work. You can apply labels to issues and pull requests to signify priority, category, or any other information you find useful. There are two types of labels - default and custom. GitHub provides default ones in every new repository. All default labels can be seen in table \ref{fig:defaultLabels} and can be used to create a standard workflow in a repository:

\begin{figure}[H]%
    \centering
	\includegraphics[width=8cm]{defaultLabels.jpg}
    \caption{Default Git labels provided for every repository}%
    \label{fig:defaultLabels}%
\end{figure}

These default labels come as a big help in directing the project and targeting the most important issues, but they don't say much about the nature of the issue itself.

The custom tags tell are used to specify the part of the project, where the issue is located but they still don't give any semantic information about the issue itself. That's the reason why this approach was rejected. 

